\item
In this problem, you will conduct a simulation study of the series-parallel  identification algorithm that has been discussed in class. To this end, use the MATLAB file {\tt sp$\_$predict.m}, which is posted on bSpace.

The system that we want to identify is the second-order system given by
\begin{align*}
    y(k) = \frac{b_1q^{-1} + b_2 q^{-2}}{1 + a_1 q^{-1} + a_2 q^{-2}}\, u(k) + w(k)
\end{align*}
where $u(k)$ is the input and $w(k)$ is the measurement noise.
The PAA uses a series-parallel model and is governed by the equations
\begin{align*}
    e^o(k+1) & =  y(k+1) - \hat{\theta}^T(k) \, \phi(k) \\
    \phi(k) & = \begin{bmatrix}
            -y(k) & -y(k-1) & u(k) & u(k-1)
        \end{bmatrix}^T \\
    \hat{\theta}(k+1) & = \hat{\theta}(k) 
        + \frac{1}{\lambda_1(k) + \phi^T(k) F(k) \phi(k)} F(k) \phi(k)\,e^0(k+1) \\
    F(k+1) & = \frac{1}{\lambda_1(k)} \, \left[ F(k) - \lambda_2(k) 
        \frac{ F(k)\phi(k)\phi^T(k)F(k) }{ \lambda_1(k) + \lambda_2(k) \phi^T(k)F(k) \phi(k) } \right] \;
\end{align*}
The plant parameters are $a_1 = 1.7$, $a_2 = 0.72$, $b_1 =0.1$ and $b_2=0.05$. (Note that $A(q^{-1})$ is anti-Schur.)

Do the following and write a summary of your findings:
\begin{itemize}
    \item
    Try the least squares adaptation gain, least square adaptation  gains with a forgetting factor, and constant adaptation gains. For the cases when you are testing constant diagonal adaptation gains
    $F = {\rm Diag}(F_{11},\,F_{22},\,F_{33},\,F_{44})$, analyze the effect on the parameter convergence of the relative magnitude between  the submatrices ${\rm Diag}(F_{11},\,F_{22})$ and ${\rm Diag}(F_{33},\,F_{44}$).  Try ratios such as 1/100, 1/10, 1/1, 10/1, etc.

    \textbf{Note:} When the adaptation gain is constant, the correction terms for $a_i$'s are proportional to  ${\rm Diag}(F_{11},\,F_{22}) \times y$, while those for the $b_i$'s are proportional to  ${\rm Diag}(F_{33},\,F_{44}) \times u$. You should see that balanced, and hence faster, convergence takes place when all correction terms are more or less of the same size.

    \item
    Verify the effect of the persistence of excitation (or lack thereof) of the input sequence $u(k)$ on the parameter error convergence.

    \item
    Check what the effect of white measurement noise $w(k)$ is on the parameter error convergence (i.e.\ turn  the noise term on and off).
\end{itemize}
