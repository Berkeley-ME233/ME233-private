\item
Define the matrices
\begin{align*}
    A_e & := \begin{bmatrix}
            A & 0 & 0 \\
            B_1 & A_1 & 0 \\
            0 & 0 & A_2
        \end{bmatrix} 
        & B_e & := \begin{bmatrix}
            B \\
            0 \\
            B_2
        \end{bmatrix} \\
    C_e & := \begin{bmatrix}
            D_1 & C_1 & 0 \\
            0 & 0 & C_2
        \end{bmatrix}
        & D_e & := \begin{bmatrix}
            0 \\
            D_2
        \end{bmatrix}
\end{align*}
For the frequency-shaped linear quadratic regulator (FSLQR) problem considered in the first half of Lecture 14, we derived the following set of conditions that guarantee the existence of the optimal FSLQR:

$\,$

\begin{tabular}{p{0.5cm}p{14cm}}
    \textbf{A1:} & $(A_e,B_e)$ is stabilizable \\
    \textbf{A2:} & The state space realization $C_e (zI - A_e)^{-1} B_e + D_e$ has no transmission zeros on the unit circle.
\end{tabular}

$\,$

It was subsequently stated (but not proved) that the following conditions imply that (A1)--(A2) hold:

$\,$

\begin{tabular}{p{0.5cm}p{14cm}}
    \textbf{B1:} 
        & $(A,B)$ is stabilizable \\
    \textbf{B2:} 
        & $A_1$ and $A_2$ are Schur \\
    \textbf{B3:}
        & nullity$\begin{bmatrix} A_2 - \lambda I & B_2 \\ C_2 & D_2 \end{bmatrix} = 0$ whenever $|\lambda| = 1$ \\
    \textbf{B4:}
        & nullity$\begin{bmatrix} A_1 - \lambda I & B_1 \\ C_1 & D_1 \end{bmatrix} = 0$ whenever $\lambda$ is a unit circle eigenvalue of $A$
\end{tabular}

$\,$

Prove that conditions (B1)--(B4) imply conditions (A1)--(A2).

\textbf{Hint:}
You will find it useful to use the following characterizations:

\begin{itemize}
    \item
    $(A,B)$ is stabilizable if and only if nullity$\begin{bmatrix} A^T - \lambda I \\ B^T \end{bmatrix} = 0$ whenever $|\lambda| \geq 1$
    
    \item
    $\lambda$ is a transmission zero of the realization $C_e (zI - A_e)^{-1} B_e + D_e$ if and only if nullity$\begin{bmatrix} A_e - \lambda I & B_e \\ C_e & D_e \end{bmatrix} > 0$
\end{itemize}

The second characterization is a result of $D_e^T D_e = D_2^T D_2 \succ 0$ holding for the FSLQR problem. 