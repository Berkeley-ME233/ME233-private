\item
We wish to determine how to split a positive number $L$ into $N$ pieces, so that the product of the $N$ pieces is maximized. The problem can be solved using dynamic programming by formulating it as follows. Consider a first-order ``pure integrator"
\begin{align*}
    x(k+1) = x(k) + u(k)\hspace{3em} x(0) = 0,
\end{align*}
we wish to determine the optimal control sequence
\begin{align*}
    U_0^o = \{ u^o(0),\,u^o(1),\,\cdots,\, u^o(N-1)\}
\end{align*}
such that:

\begin{enumerate}
    \item
    $u^o(k) \ge 0$.

    \item
    $x(N) = L$.

    \item
    The following cost function is maximized:
    \begin{align*}
        J  = \prod_{k=0}^{N-1}\, u (k) = u (0) \,u (1) \,\cdots \, u(N-1)
    \end{align*}
\end{enumerate}

To use dynamic programming, it is convenient to define the following optimal value function
\begin{align*}
    J_m^o[x(m)]  = \max_{U_m} \prod_{k=m}^{N-1}\, u(k)
\end{align*}
where $U_m  = \{ u(m),\,u(m+1),\,\cdots,\, u(N-1)\}$ is the set of all  feasible control sequences from the instance $m$.

\textbf{Hint:} Notice that, because of the terminal condition $x(N) = L$, and the state equation, the optimal value function at $x(N-1)$ is given by
\begin{align*}
    J^o[x(N-1)] = u^o(N-1) = L - x(N-1)
\end{align*}
Use the Bellman equation starting from this boundary condition.
