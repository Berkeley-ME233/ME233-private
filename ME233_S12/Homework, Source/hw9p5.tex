\item
In problem 3, you conducted a simulation study of the series-parallel identification algorithm that has been discussed in class. In this problem you are asked to conduct a similar simulation study of the parallel identification algorithm that you analyzed in problem 4.

The system that we want to identify is the same as the one in problem 3, i.e.
\begin{align*}
    y(k) = \frac{b_1q^{-1} + b_2 q^{-2}}{1 + a_1 q^{-1} + a_2 q^{-2}}\, u(k) + w(k)
\end{align*}
where $u(k)$ is the input and $w(k)$ is measurement noise. The plant parameters are $a_1 = 1.7$, $a_2 = 0.72$, $b_1 =0.1$ and $b_2=0.05$.

\begin{enumerate}
    \item
    Select $\bar{\lambda} \leq 1$ and a set of constants $c_1$ and $c_2$ so that:
    \begin{itemize}
        \item $c_i \neq a_i$, $i=1,2$ and
        \item $G(z^{-1})$ in \eqref{eq:gq} satisfies the conditions that you determined in problem 2(c).
    \end{itemize}

    \item
    Using the matlab file {\tt sp$\_$predict.m} as your starting point, create a MATLAB file
    %{\tt p$\_$predict$\_${\em your-last-name}.m}
    that implements the parallel RLS PAA in Eqs.~\eqref{eq:parallel_PAA_first}--\eqref{eq:parallel_PAA_last}. Compare the parameter convergence of the parallel algorithm with that of the series-parallel algorithm, using similar initial conditions, under the following cases:
    \begin{enumerate}
        \item
        Random input $u(k)$, no measurement noise, and least square gain (without forgetting factor).

        \item
        Random input $u(k)$, no measurement noise, and least square gain with forgetting factor.
        \item
        Random input $u(k)$, white measurement noise, and least square gain (without forgetting factor).

        \item
        Random input $u(k)$, colored measurement noise, and least square gain (without forgetting factor).
    \end{enumerate}
    Write a short summary of your findings. (You do not need to submit your modified version of {\tt sp$\_$predict.m})

%    \textbf{Note:} To perform the simulations for the parallel RLS PAA, it will be easiest to create a new function based on {\texttt sp$\_$predict.m}.
\end{enumerate}


